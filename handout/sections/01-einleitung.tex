\section{Einleitung}

Große Computernetzwerke, allen voran das Internet,
setzen sich zahlreichen Bestandteilen zusammen,
die von vielen verschiedenen unabhängigen Dienstleistern betrieben werden.
Da diese in erster Linie von Eigeninteressen geleitet werden,
ist die Spieltheorie ein naheliegender Ansatz
zur Untersuchung der Entstehung solcher Netzwerke.

In Abschnitt \ref{sec:local} betrachten wir das \emph{Local Connection Game},
bei dem die Spieler zwei konkurrierende Ziele verfolgen.
Einerseits gilt es, den eigenen Knoten möglichst effizient
an die Knoten der üblichen Spieler anzuschließen,
andererseits soll dies unter minimalem finanziellen Eigenaufwand geschehen.
Zur Präzisierung der Effizienz werden dabei
die Pfadlängen im entstehenden Netzwerk
(vergleiche Definition \ref{def:dist}) verwendet.
Dieses Spiel modelliert etwa die Aufgabe,
Netze verschiedener Betreiber untereinander zu verbinden.

Abschnitt \ref{sec:global} behandelt das \emph{Global Connection Game}.
Hier besteht die Aufgabe eines Spielers darin,
zwei verschiedene Knoten möglichst preisgünstig zu verbinden.
Die Kosten einer Kante im entstehenden Netzwerk
werden dabei unter allen Spielern fair aufgeteilt,
die diese Kante für ihre Zwecke nutzen.

Zur Bewertung der entstehenden Netzwerke
nutzen wir jeweils die Gemeinwohlkosten,
also die Summe der Kosten der beteiligten Akteure.
Da beide betrachteten Spiele nur endlich viele Strategievektoren
sowie Nash-Gleichgewichte besitzen,
lassen sich stets der Preis der Stabilität und der Preis der Anarchie
(siehe Definition \ref{def:pds_pda}) bilden.
Diese beiden Größen beschreiben den Unterschied
zwischen der Qualität eines Netzwerkes im Nash-Gleichgewicht
und einer optimalen Lösung hinsichtlich der individuellen Ziele.

\begin{definition}
  Die Nash-Gleichgewichte eines Spiels bezeichnen wir auch
  als \emph{stabile} Strategievektoren.
\end{definition}

\begin{definition}
  Eine \emph{Kostenfunktion} für ein Spiel
  ist eine Funktion
  $\mathrm{cost} \! \mathcolon \symcal{S} \rightarrow \symbb{R}_{ \geq 0 }$
  von der Menge aller Strategievektoren $\symcal{S}$
  in die nicht-negativen reellen Zahlen.
  Ein Strategievektor $M$ heißt \emph{optimal},
  falls $\cost{M}$ minimal im Bild von $\mathrm{cost}$ ist.
\end{definition}

\begin{definition}
\label{def:pds_pda}
  Sei $\mathrm{cost}$ die Kostenfunktion eines Spiels
  und $\symcal{N}$ die Menge aller Nash-Gleichgewichte.
  Wir definieren den \emph{Preis der Stabilität}
  und den \emph{Preis der Anarchie} dieses Spiels bezüglich $\mathrm{cost}$ als
  \[
    \text{Preis der Stabilität}
    \coloneq
    \frac{\min_{ N \in \symcal{N} } \cost{ N }}{\cost{M}}
    \quad
    \text{Preis der Anarchie}
    \coloneq
    \frac{\max_{ N \in \symcal{N} } \cost{ N }}{\cost{M}}
  \]
  falls sowohl ein optimaler Strategievektor $M$
  als auch das verwendete Minimum bzw.\ Maximum existiert.
\end{definition}

\begin{definition}
\label{def:dist}
  Sei $G \equal \lparen V \mathcomma E \rparen$ ein Graph
  mit $\vert V \vert \geq 2$.
  Sind zwei Knoten $u \ne v$ über einen Pfad in $G$ verbunden,
  bezeichnet $\dist{u}{v}$ die Länge eines minimalen derartigen Pfades.
  Existiert kein Pfad zwischen $u$ und $v$,
  setzen wir $\dist{u}{v} \equal \infty$.
  Außerdem heißt~$\max_{ u \ne v } \dist{u}{v}$
  der \emph{Durchmesser} von $G$.
\end{definition}
