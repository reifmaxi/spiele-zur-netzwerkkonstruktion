\section{Das Local Connection Game}
\label{sec:local}

Das \emph{Local Connection Game} ist ein Spiel zur Netzwerkkonstruktion
für mindestens zwei Spieler.
Jedem der $n$ Spieler ist ein eigener Knoten $u_i$
aus einer $n$-elementigen Knotenmenge $V$ zugeteilt
und das Ziel ist es,
die Pfadlängen von dem eigenen zu den anderen Knoten
unter möglichst geringen Ausgaben zu minimieren.
Dafür können Kanten zu einem festgelegten Preis
$\mupalpha \in \lparen 0 \mathcomma \infty \rparen$
gekauft werden,
die gegebenenfalls von allen Akteuren genutzt werden dürfen.
Eine Strategie des Spielers $i$ ist also
eine Menge $S_i$ von zum Knoten $u_i$ adjazenten ungerichteten Kanten
und jeder Strategievektor
$S \equal \lparen S_1 \mathcomma \unicodeellipsis \mathcomma S_n \rparen$
führt zu einem Graphen
$G \lparen S \rparen \coloneq \lparen V \mathcomma E_S \rparen$
mit $E_S \coloneq \bigcup_{ i\equal 1 }^n S_i$.
Die unter $S$ entstehenden Kosten für Spieler $i$ sind durch
\[
  c_i \lparen S \rparen
  \coloneq
  \mupalpha \vert S_i \vert
  \mathplus \sum_{ v \in V \setminus \lbrace u_i \rbrace } \dist{u_i}{v}
\]
gegeben.
Diese Wahl der Kostenfunktion vereint die beiden Ziele der Spieler,
einerseits die Qualität der eigenen Anbindung zu steigern,
andererseits den dafür notwendigen finanziellen Aufwand
auf ein Minimum zu beschränken.
Schließlich entsprechen die Gemeinwohlkosten unter dem Strategievektor $S$
der Summe der unter $S$ entstehenden individuellen Kosten,
also
\[
  \SC{S}
  \coloneq
  \sum_{ i \equal 1 }^n c_i \lparen S \rparen
  \equal
  \mupalpha \sum_{ i \equal 1 }^n \vert S_i \vert
  \mathplus \sum_{ u \ne v } \dist{u}{v}.
\]

Existiert einerseits zwischen zwei Knoten kein Pfad,
so sind die Kosten der dazugehörigen Spieler
und damit auch die Gemeinwohlkosten unendlich.
Ist andererseits eine Kante in Strategien von mehreren Spielern enthalten,
so lassen sich sowohl individuelle als auch die Gemeinwohlkosten senken,
indem nur ein Spieler besagte Kante in seiner Strategie behält.
Somit bilden sowohl stabile als auch optimale Strategievektoren
stets zusammenhängende einfache Graphen.

\bigskip

Zunächst untersuchen wir den Preis der Stabilität des Local Connection Game
bezüglich der Gemeinwohlkosten.
Dazu bestimmen wir in Abhängigkeit von $\mupalpha$
optimale Lösungen und zeigen anschließend,
für welche Werte von $\mupalpha$ diese auch Nash-Gleichgewichte bilden.

\begin{lemma}
\label{la:local_optimal}
  Seien $R$ und $T$ Strategievektoren im Local Connection Game,
  sodass $G \lparen R \rparen$ sternförmig
  und $G \lparen T \rparen$ vollständig ist.
  Dann ist $R$ optimal falls $\mupalpha \geq 2$,
  und $T$ falls $\mupalpha \leq 2$.
\end{lemma}

\begin{proof}
  Sei $\mupalpha \geq 2$
  sowie $M_\geq$ ein diesbezüglich optimaler Strategievektor
  und $m \coloneq \vert E_{ M_\geq } \vert$.
  Zunächst schätzen wir $\SC{M_\geq}$ nach unten ab,
  wofür wir alle $n \lparen n \minus 1 \rparen$ Knotenpaare betrachten müssen.
  Die Knoten von $2m$ solcher Tupel sind direkt miteinander verbunden;
  die Knoten der restlichen
  $n \lparen n \minus 1 \rparen \minus 2m$ Tupel
  haben mindestens Abstand $2$.
  Es gilt also
  \[
    \SC{ M_\geq }
    \geq
    \mupalpha m \mathplus 2m
    \mathplus 2 \cdot \lparen n \lparen n \minus 1 \rparen
    \minus 2m \rparen
    \equal
    \lparen \mupalpha \minus 2 \rparen m
    \mathplus 2n \lparen n \minus 1 \rparen.
  \]
  Da einerseits $G \lparen M_\geq \rparen$ zusammenhängend
  und andererseits die gefundene untere Schranke monoton steigend in $m$ ist,
  nimmt diese für $m \equal n \minus 1$
  den kleinsten in Frage kommenden Wert an.
  Nach Voraussetzung ist $M_\geq$ optimal,
  weswegen $\SC{ M_\geq }$ genau diesem minimalen Wert entspricht.
  Das bedeutet jedoch,
  dass alle Knoten in $G \lparen M_\geq \rparen$ nicht nur mindestens,
  sondern auch höchstens Abstand $2$ haben.
  Damit ist $M_\geq$ ein Sterngraph folglich $R$ ebenfalls optimal.

  Sei nun $\mupalpha \leq 2$
  und $M_\leq$ ein in diesem Falle optimaler Strategievektor.
  Mit $k \coloneq \vert E_{ M_\leq } \vert$ erhält man ebenso wie oben
  \begin{equation}
    \label{eq:local_optimal}
    \SC{ M_\leq }
    \geq
    \lparen \mupalpha \minus 2 \rparen k
    \mathplus 2n \lparen n \minus 1 \rparen
  \end{equation}
  und gleichermaßen entspricht $\SC{ M_\leq }$
  dem kleinsten sinnvollen Wert
  der unteren Schranke aus (\ref{eq:local_optimal}).
  Wegen $\mupalpha \leq 2$ ist diese nun monoton fallend in $k$
  und $G \lparen M_\leq \rparen$ folglich vollständig.
  Also ist auch $T$ ein optimaler Strategievektor.
\end{proof}

\begin{figure}[h]
  \centering
  \caption{Zwei mögliche Graphen bei sechs Spielern.}
  \label{im:sn_kn}
  \subcaptionbox{$G \lparen R \rparen$ aus Lemma \ref{la:local_stabil_stern}.}{%
    \includegraphics[width=0.45\textwidth]{s5}
  }
  \quad
  \subcaptionbox{Eine Option für $G \lparen T \rparen$ aus Lemma
    \ref{la:local_stabil_vollständig}.}{%
    \includegraphics[width=0.45\textwidth]{k6}
  }
\end{figure}

\begin{lemma}
\label{la:local_stabil_stern}
  Sei $R$ ein Strategievektor
  im Local Connection Game mit mindestens drei Spielern,
  sodass $G \lparen R \rparen$ sternförmig ist.
  Es ist $R$ genau dann ein Nash-Gleichgewicht,
  wenn $\mupalpha \geq 1$.
\end{lemma}

\begin{proof}
  Wir zeigen die Behauptung für den Strategievektor
  $R \equal \lparen R_1 \mathcomma
  \varnothing \mathcomma \unicodeellipsis \mathcomma \varnothing \rparen$,
  bei dem Spieler $1$ alle zum Knoten $u_1$ adjazenten Kanten kauft.
  Abbildung \ref{im:sn_kn} zeigt $G \lparen R \rparen$ für sieben Spieler.
  Für andere Strategievektoren mit sternförmigem Graphen
  erhält man durch analoge Beobachtungen dasselbe Resultat.

  Sei also $\mupalpha \geq 1$.
  Spieler $1$ kann seine Strategie nur durch Abstoßen einer Kante
  $\lbrace u_1 \mathcomma v \rbrace$ ändern.
  Weil daraus aber $\dist{u_1}{v} \equal \infty$ folgt,
  ist diese Strategieänderung nicht rentabel.
  Jeder andere Spieler kann seine Strategie
  nur durch Hinzufügen von $k$ Kanten ändern.
  Damit verringert er einerseits die für ihn relevanten Pfadlängen
  um höchstens~$k$,
  erhöht aber gleichzeitig seine finanziellen Ausgaben
  um $\mupalpha k \geq k$.
  Es gibt also für keinen Spieler einen vorteilhaften Strategiewechsel.

  Gelte nun $\mupalpha \less 1$.
  Es gibt mindestens drei Spieler,
  folglich existieren $i$ und $j$ mit $i \ne j$
  und $R_i \equal \varnothing \equal R_j$.
  Entscheidet sich Spieler $i$ dazu,
  seiner Strategie die Kante $\lbrace u_i \mathcomma u_j \rbrace$ hinzuzufügen,
  verringert das $\dist{u_i}{u_j}$ um $1$
  und kostet lediglich $\mupalpha \less 1$.
  Dies ist also ein sinnvoller Strategiewechsel.
\end{proof}

\begin{lemma}
\label{la:local_stabil_vollständig}
  Sei $T$ ein Strategievektor
  im Local Connection Game mit mindestens drei Spielern,
  sodass $G \lparen T \rparen$ vollständig ist.
  Es ist $T$ genau dann ein Nash-Gleichgewicht,
  wenn~$\mupalpha \leq 1$.
\end{lemma}

\begin{proof}
  Sei zunächst $\mupalpha \leq 1$.
  Kein Spieler kann durch Aufnahme einer Kante in seine Strategie
  einen Vorteil erlangen,
  denn die Pfadlängen zwischen allen Knoten in $G \lparen T \rparen$
  sind bereits minimal.
  Entfernt ein Spieler $i$ aus seiner Strategie $k$ Kanten,
  verringert er so seine finanziellen Kosten um $\mupalpha k$;
  gleichzeitig erhöhen sich dadurch aber auch die Pfadlängen
  von $u_i$ zu den restlichen Knoten um mindestens $k \geq \mupalpha k$.
  Somit kann kein Spieler seine Kosten durch eine Strategieänderung verringern.

  Sei nun $\mupalpha \greater 1$.
  Da $G \lparen T \rparen$ vollständig ist,
  existiert ein Spieler $i$ mit nicht-leerer Strategie.
  Das Abstoßen einer beliebiger Kante $\lbrace u_i \mathcomma v \rbrace \in T_i$
  erhöht $\dist{u_i}{v}$ um lediglich $1$,
  denn es gibt mindestens drei Spieler (vergleiche Abbildung \ref{im:sn_kn}).
  Weil so aber $\mupalpha \greater 1$ finanzielle Kosten gespart werden,
  ist diese Strategieänderung vorteilhaft für Spieler~$i$.
\end{proof}

\begin{bemerkung}
  Wie aus Abbildung \ref{im:k2} ersichtlich,
  gibt es für $n \equal 2$ keinen Unterschied
  zwischen dem Sterngraph und dem vollständigen Graphen.
  Dieser entsteht unabhängig von $\mupalpha$
  aus allen in diesem Fall optimalen und stabilen Strategievektoren.
  \begin{figure}[h]
    \centering
    \includegraphics[angle=90,height=1cm]{k2}
    \caption{Der vollständige Graph mit zwei Knoten.}
    \label{im:k2}
  \end{figure}
\end{bemerkung}

\begin{satz}
  Der Preis der Stabilität des Local Connection Game
  ist höchstens $\frac{4}{3}$,
  und genau~$1$
  falls $\mupalpha \in \lparen 1 \mathcomma 2 \rparen ^\mathrm{c}$.
\end{satz}

\begin{proof}
  Aus den Lemmas \ref{la:local_optimal}, \ref{la:local_stabil_stern}
  und \ref{la:local_stabil_vollständig}
  sowie der anschließenden Bemerkung ist ersichtlich,
  dass für $\mupalpha \in \lparen 1 \mathcomma 2 \rparen ^\mathrm{c}$
  optimale Nash-Gleichgewichte existieren.
  Daher ist in diesem Fall der Preis der Stabilität $1$.

  Sei nun $\mupalpha \in \lparen 1 \mathcomma 2 \rparen$.
  Gemäß Lemma \ref{la:local_optimal} ist der Strategievektor
  $R \equal \lparen R_1 \mathcomma
  \varnothing \mathcomma \unicodeellipsis \mathcomma \varnothing \rparen$,
  bei dem Spieler $1$ alle zum Knoten $u_1$ adjazenten Kanten kauft,
  optimal.
  Für diesen ergeben sich dabei Kosten in Höhe von
  $\mupalpha \lparen n \minus 1 \rparen \mathplus \lparen n \minus 1 \rparen$.
  Die anderen Spieler haben keine finanziellen Ausgaben
  und untereinander Abstand $2$.
  Unter Berücksichtigung des Abstands zu Spieler~$1$
  führt dies jeweils zu Kosten von $2 \lparen n \minus 2 \rparen \mathplus 1$.
  Folglich gilt
  \[
    \SC{R}
    \equal
    \mupalpha \lparen n \minus 1 \rparen \mathplus \lparen n \minus 1 \rparen
    \mathplus
    \lparen n \minus 1 \rparen
    \lparen 2 \lparen n \minus 2 \rparen \mathplus 1 \rparen
    \equal
    \lparen n \minus 1 \rparen
    \lparen \mupalpha \mathplus 2n \minus 2 \rparen.
  \]
  Um den Preis der Stabilität nach oben abzuschätzen reicht es aus,
  die Gemeinwohlkosten eines beliebigen Nash-Gleichgewichts zu betrachten.
  Lemma \ref{la:local_stabil_vollständig} zufolge wählen wir $T$
  mit vollständigem $G \lparen T \rparen$
  und erhalten mit (\ref{eq:local_optimal})
  aus dem Beweis von Lemma \ref{eq:local_optimal}
  \[
    \SC{T}
    \equal
    \lparen \mupalpha \minus 2 \rparen
    \frac{n \lparen n \minus 1 \rparen}{2}
    \mathplus 2 n \lparen n \minus 1 \rparen
    \equal
    \lparen n \minus 1 \rparen
    \left \lparen \frac{\mupalpha n}{2} \minus n \mathplus 2n \right \rparen
    \equal
    \frac{1}{2}
    \lparen n \minus 1 \rparen
    \lparen \mupalpha n \mathplus 2n \rparen.
  \]
  Damit ist
  \[
    \frac{\SC{R}}{\SC{T}}
    \equal
    \frac{\lparen n \minus 1 \rparen
    \lparen \mupalpha \mathplus 2n \minus 2 \rparen}
    {\frac{1}{2} \lparen n \minus 1 \rparen
    \lparen \mupalpha n \mathplus 2n \rparen}
    \equal
    \frac{2 \mupalpha \mathplus 4n \minus 4}
    {\lparen \mupalpha \mathplus 2 \rparen n}
  \]
  eine obere Schranke für den Preis der Stabilität.
  Fasst man diesen Quotienten als Funktion in $\mupalpha$ auf,
  ist diese durch $n \geq 2$ auf ganz $\symbb{R}_{ \greater 0 }$
  monoton fallend.
  Wegen $\mupalpha \in \lparen 1 \mathcomma 2 \rparen$ gilt daher
  \[
    \frac{\SC{R}}{\SC{T}}
    \equal
    \frac{2 \mupalpha \mathplus 4n \minus 4}
    {\lparen \mupalpha \mathplus 2 \rparen n}
    \leq
    \frac{2 \mathplus 4n \minus 4}{\lparen 1 \mathplus 2 \rparen n}
    \equal
    \frac{4n \minus 2}{3n}
    \less
    \frac{4}{3}.
  \]
\end{proof}

\bigskip

Im Folgenden untersuchen wir das Local Connection Game
im Hinblick auf den Preis der Anarchie bezüglich der Gemeinwohlkosten.
Dazu zitieren wir ein Resultat,
das die Gemeinwohlkosten von Nash-Gleichgewichten
in Abhängigkeit ihrer Durchmesser beschränkt,
und finden anschließend eine obere Schranke für eben diese Durchmesser
in Abhängigkeit von $\mupalpha$.

\begin{lemma}
  \label{la:local_durchmesser1}
  Sei $N$ ein Nash-Gleichgewicht im Local Connection Game
  und $d$ der Durchmesser von $G \lparen N \rparen$.
  Dann gilt $\SC{N} \leq O \lparen d \rparen \cdotp \SC{M}$
  für einen beliebigen optimalen Strategievektor $M$.
\end{lemma}

\begin{proof}
  Dies ist Lemma 19.4 dem Buch \emph{Algorithmic Game Theory}
  \cite[S.~491f]{tardos_wexler_2007}.
\end{proof}

\begin{lemma}
\label{la:local_durchmesser2}
  Sei $N$ ein Nash-Gleichgewicht im Local Connection Game
  und $d \geq 2$ der Durchmesser von $G \lparen N \rparen$.
  Dann gilt $d \leq 2 \sqrt{\mupalpha}$.
\end{lemma}

\begin{proof}
  Seien $u$ und $v$ zwei Knoten in $G \lparen N \rparen$
  mit $\dist{u}{v} \equal d \geq 2$.
  Gemäß Lemma \ref{la:local_stabil_vollständig} ist~$\mupalpha \geq 1$
  und daher $2 \sqrt{\mupalpha} \geq 2$ ein sinnvoller Kandidat
  für eine obere Schranke von $d$.
  Dass dafür die Voraussetzung $d \geq 2$ tatsächlich nötig ist,
  zeigt anschließende Bemerkung.

  Wir betrachten nun den Fall,
  dass $\mathrm{dist} \lparen u \mathcomma v \rparen$ gerade
  mit $\mathrm{dist} \lparen u \mathcomma v \rparen \equal 2k$ ist.
  Ein kürzester Pfad zwischen $u$ und $v$ ist dann von der Form
  $P \equal \lparen u \mathcomma w_1 \mathcomma \unicodeellipsis
  \mathcomma w_{ 2k \minus 1 } \mathcomma v \rparen$.

  \begin{figure}[h]
    \centering
    \includegraphics[width=0.85\textwidth]{pfad_gerade}
    \caption{Der Pfad $P$
      mit zusätzlicher Kante $\lbrace u \mathcomma v \rbrace$.}
    \label{im:pfad_gerade}
  \end{figure}

  Aufgrund der Minimalität von $P$ ist für jeden Knoten $w_i$ der Teilpfad
  $\lparen u \mathcomma w_1 \mathcomma
  \unicodeellipsis \mathcomma w_i \rparen$
  ein minimaler Pfad zwischen $u$ und $w_i$,
  und so gilt $\mathrm{dist} \lparen u \mathcomma w_i \rparen \equal i$.
  Erweitert man $G \lparen N \rparen$
  um die Kante $\lbrace u \mathcomma v \rbrace$
  wie in Abbildung \ref{im:pfad_gerade},
  so erhält man neben $\mathrm{dist} \lparen u \mathcomma v \rparen \equal~1$
  auch neue Pfade
  $\lparen u \mathcomma v \mathcomma w_{ 2k \minus 1 } \mathcomma
  \unicodeellipsis \mathcomma w_i \rparen$
  der Länge $2k \minus i \mathplus 1$.
  Für alle $w_i$ mit $i \geq k \mathplus 1$ sind diese wegen
  \[
    2k \minus i \mathplus 1
    \leq
    2 \lparen i \minus 1 \rparen \minus i \mathplus 1
    \equal
    i \minus 1
    \less
    i
  \]
  kürzer als $\mathrm{dist} \lparen u \mathcomma w_i \rparen$
  im ursprünglichen Graphen.
  Insgesamt verringern sich daher auf diese Weise
  die für Knoten $u$ relevanten Pfadlängen um
  \[
    2k \minus 1
    \mathplus \sum_{ i \equal k \mathplus 1 }^{ 2k \minus 1 }
    \Big \lparen i \minus \lparen 2k \minus i \mathplus 1 \rparen \Big \rparen
    \equal
    \sum_{ i \equal k \mathplus 1 }^{ 2k }
    \Big \lparen i \minus \lparen 2k \minus i \mathplus 1 \rparen \Big \rparen
    \equal
    \sum_{ i \equal 1 }^k \lparen 2i \minus 1 \rparen
    \equal
    k^2.
  \]

  Ist $\mathrm{dist} \lparen u \mathcomma v \rparen$ hingegen ungerade
  mit $\mathrm{dist} \lparen u \mathcomma v \rparen \equal 2k \minus 1$
  für ein $k \geq 2$,
  so umfasst ein kürzester Pfad zwischen $u$ und $v$ genau
  $2k \minus 2$ weitere Knoten,
  etwa $P \equal \lparen u \mathcomma w_1 \mathcomma \unicodeellipsis
  \mathcomma w_{ 2k \minus 2 } \mathcomma v \rparen$.
  Gleichermaßen erhält man
  durch Hinzufügen der Kante $\lbrace u \mathcomma v \rbrace$
  für alle $w_i$ mit $i \geq k \mathplus 1$
  neue kürzeste Pfade von $u$ zu $w_i$,
  welche in diesem Fall jeweils Länge $2k \minus i$ haben.
  Die für $u$ relevanten Pfadlängen verringern sich dadurch also um insgesamt
  \[
    \lparen 2k \minus 1 \rparen \minus 1
    \mathplus \sum_{ i \equal k \mathplus 1 }^{ 2k \minus 2 }
    \Big \lparen i \minus \lparen 2k \minus i \rparen \Big \rparen
    \equal
    \sum_{ i \equal k \mathplus 1 }^{ 2k \minus 1 }
    \Big \lparen i \minus \lparen 2k \minus i \rparen \Big \rparen
    \equal
    \sum_{ i \equal 1 }^{ k \minus 1 } 2i
    \equal
    k^2.
  \]

  Ist nun $k \greater \sqrt{\mupalpha}$,
  so erhält der zum Knoten $u$ gehörige Spieler folglich stets einen Vorteil,
  wenn er seine bisherige Strategie
  um die Kante $\lbrace u \mathcomma v \rbrace$ ergänzt.
  Da $N$ aber ein Nash-Gleichgewicht ist,
  kann das nicht sein.
  Somit gilt $k \leq \sqrt{\mupalpha}$
  und damit schließlich auch~${d \equal
  \mathrm{dist} \lparen u \mathcomma v \rparen \leq 2 \sqrt{\mupalpha}}$.
\end{proof}

\begin{bemerkung}
  Sei $N$ ein Nash-Gleichgewicht im Local Connection Game,
  sodass $G \lparen N \rparen$ Durchmesser $d \equal 1$ hat.
  Dann ist $G \lparen N \rparen$ vollständig und
  Lemma \ref{la:local_stabil_vollständig} liefert folglich $\mupalpha \leq 1$.
  Anders als im vorangegangenen Beweis können wir also nicht folgern,
  dass $2 \sqrt{\mupalpha}$ eine obere Schranke für $d$ ist.
  Im Gegenteil, für $\mupalpha \less \frac{1}{4}$ gilt
  $\sqrt{\mupalpha} \less \frac{1}{2}$
  und so ist $2 \sqrt{\mupalpha} \less 1 \equal d$.
\end{bemerkung}

\begin{satz}
  Der Preis der Anarchie des Local Connection Game ist höchstens
  $O \lparen \sqrt{ \mupalpha } \rparen$.
\end{satz}

\begin{proof}
  Dies folgt sofort
  aus den Lemmas \ref{la:local_durchmesser1} und \ref{la:local_durchmesser2}
  sowie der nachfolgenden Bemerkung.
\end{proof}
