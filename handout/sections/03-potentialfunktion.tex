\section{Spiele mit exakter Potentialfunktion}
\label{sec:potential}

In diesem Abschnitt führen wir \emph{exakte Potentialfunktionen} ein,
mit deren Hilfe sich der Preis der Stabilität beschränken lässt.
Dieses Vorgehen ist auch als \emph{Potentialfunktion-Methode} bekannt
und findet im folgenden Abschnitt \ref{sec:global} Anwendung.

\begin{definition}
  Seien $c_i \! \mathcolon \symcal{S} \rightarrow \symbb{R}$
  die Kostenfunktionen in einem Spiel für $k$ Spieler
  und bezeichne $\symcal{S}$ die Menge aller Strategievektoren.
  Eine Funktion
  $\mupPhi \! \mathcolon \symcal{S} \rightarrow \symbb{R}$
  heißt \emph{gewichtete Potentialfunktion} dieses Spieles,
  falls es $w_i \in \lparen 0 \mathcomma \infty \rparen$ gibt,
  sodass für alle $S \mathcomma S \prime \in \symcal{S}$
  \[
    \mupPhi \lparen S \rparen \minus \mupPhi \lparen S \prime \rparen
    \equal
    w_i \cdotp \big \lparen c_i \lparen S \rparen
    \minus c_i \lparen S \prime \rparen \big \rparen
  \]
  gilt,
  sofern $S \prime$ durch einen Strategiewechsel von Spieler $i$
  aus $S$ hervorgeht.
  Eine gewichtete Potentialfunktion mit $w_i \equal 1$ für alle $i$
  heißt \emph{exakte Potentialfunktion}.
\end{definition}

\begin{satz}
\label{thm:pot_nash}
  Sei $\mupPhi$ eine exakte Potentialfunktion eines Spiels
  und $N$ ein Strategievektor,
  für den $\mupPhi \lparen N \rparen$ minimal ist.
  Dann ist $N$ ein Nash-Gleichgewicht.
\end{satz}

\begin{proof}
  Jeder beliebige Strategiewechsel eines Spielers $i$ von $N_i$ zu $S_i$
  ergibt einen neuen Strategievektor
  $S \equal \lparen S_i \mathcomma N_{ \minus i } \rparen$.
  Nach Voraussetzung gilt
  $\mupPhi \lparen S \rparen \geq \mupPhi \lparen N \rparen$
  und so folgt
  \[
    c_i \lparen S \rparen \minus c_i \lparen N \rparen
    \equal
    \mupPhi \lparen S \rparen \minus \mupPhi \lparen N \rparen
    \geq
    0.
  \]
  Also ist auch $c_i \lparen S \rparen \geq c_i \lparen N \rparen$
  und so der Strategiewechsel für Spieler $i$ nicht rentabel.
  Weil $i$ beliebig war, ist $N$ folglich ein Nash-Gleichgewicht.
\end{proof}

\begin{satz}
\label{thm:potential_bounds}
  Sei $\mupPhi$ eine exakte Potentialfunktion eines Spiels mit Kostenfunktion
  $\mathrm{cost}$.
  Gibt es $A \mathcomma B \in \lparen 0 \mathcomma \infty \rparen$ mit
  \[
    \frac{\cost{S}}{A}
    \leq
    \mupPhi \lparen S \rparen
    \leq
    B \cdotp \cost{S}
  \]
  für alle Strategievektoren $S$,
  sowie einen Strategievektor $N$, der $\mupPhi$ minimiert,
  dann ist $AB$ eine obere Schranke für den Preis der Stabilität dieses Spiels
  bezüglich $\mathrm{cost}$.
\end{satz}

\begin{proof}
  Seien $A \mathcomma B \in \lparen 0 \mathcomma \infty \rparen$ wie gefordert
  und $M$ und $N$ Strategievektoren,
  sodass $M$ optimal und $\mupPhi \lparen N \rparen$ minimal ist.
  Aus diesen Voraussetzungen folgt sofort
  \begin{equation}
  \label{eq:potential_bounds}
    \frac{\cost{N}}{A}
    \leq
    \mupPhi \lparen N \rparen
    \leq
    \mupPhi \lparen M \rparen
    \leq
    B \cdotp \cost{M}.
  \end{equation}
  Weil $N$ gemäß Satz \ref{thm:pot_nash} ein Nash-Gleichgewicht ist,
  bildet $\frac{\cost{N}}{\cost{M}}$ eine obere Schranke
  für den Preis der Stabilität dieses Spieles bezüglich $\mathrm{cost}$.
  Diese ist wiederum nach (\ref{eq:potential_bounds})
  selbst durch $AB$ nach oben beschränkt,
  woraus die Behauptung folgt.
\end{proof}
