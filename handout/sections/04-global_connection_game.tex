\section{Das Global Connection Game}
\label{sec:global}

Im \emph{Global Connection Game} ist
eine $n$-elementige Knotenmenge $V$ gegeben,
aus der jedem der $k$ Spieler zwei Knoten $s_i$ und $t_i$ zugeordnet werden.
Ziel eines Spieler~$i$ ist es,
die Knoten $s_i$ und $t_i$ unter möglichst geringem Kostenaufwand zu verbinden.
Eine Strategie $S_i$ ist also ein (gerichteter) Pfad von $s_i$ nach $t_i$,
und aus jedem Strategievektor~$S$ resultiert ein Graph
$G \lparen S \rparen \coloneq \lparen V \mathcomma E_S \rparen$.
Dabei ist $E_S$ die Menge aller Kanten,
die in den individuellen Strategien von $S$ vorkommen.
Jede in Frage kommende Kante $e$ verursacht Kosten in Höhe von
$c_e \in \lparen 0 \mathcomma \infty \rparen$,
und falls $e \in E_S$ bezeichnet $k_e \lparen S \rparen$ die Anzahl an Spielern,
deren individuelle Strategie $e$ beinhaltet.
Die Kosten für Spieler $i$ unter der Strategie $S$ sind gegeben durch
\[
  c_i \lparen S \rparen
  \coloneq
  \sum_{ e \in S_i } \frac{c_e}{k_e \lparen S \rparen},
\]
das heißt $c_e$ wird fair unter allen Spielern aufgeteilt,
deren gewählter Pfad über $e$ verläuft.
Die Gemeinwohlkosten unter $S$ sind
die Summe der individuellen Kosten,
also
\[
  \SC{S}
  \coloneq
  \sum_{ i \equal 1 }^k c_i \lparen S \rparen
  \equal \sum_{ e \in E_s } c_e.
\]
Die behauptete Gleichung lässt sich leicht
durch Umordnen der auftretenden Summanden zeigen,
und so sieht man, dass $\SC{S}$
genau den zum Errichten von $G \lparen S \rparen$ nötigen Kosten entspricht.

\begin{figure}[h]
  \centering
  \includegraphics[width=0.48\textwidth]{strategie_r}
  \hfill
  \includegraphics[width=0.48\textwidth]{strategie_t}
  \caption{Zwei mögliche Strategien bei $k \equal 3$ und $n \equal 12$.}
  \label{im:global}
\end{figure}

\medskip

Wir beginnen damit,
den Preis der Stabilität des Global Connection Game zu untersuchen.
Dazu finden wir eine exakte Potentialfunktion
und wenden die Resultate aus Abschnitt \ref{sec:potential} an.

\begin{definition}
\label{def_psi}
  Sei $S$ eine Strategie im Global Connection Game.
  Für jede Kante $e \in E_S$ definieren wir
  \[
    \mupPsi_e \lparen S \rparen
    \coloneq
    c_e \cdotp H_{ k_e \lparen S \rparen },
  \]
  wobei
  $H_k \equal \sum_{ i \equal 1 }^k \frac{1}{i}$
  die $k$-te harmonische Zahl bezeichnet.
  Zusätzlich sei
  \[
    \mupPsi \lparen S \rparen
    \coloneq
    \sum_{ e \in E_S } \mupPsi_e \lparen S \rparen.
  \]
\end{definition}

\begin{lemma}
\label{la:global_potential}
  Die Funktion $\mupPsi$ aus Definition \ref{def_psi}
  ist eine exakte Potentialfunktion für das Global Connection Game.
\end{lemma}

\begin{proof}
  Betrachte einen Strategievektor
  $R \equal \lparen R_1 \mathcomma \unicodeellipsis \mathcomma R_k \rparen$
  für das Global Connection Game
  sowie eine andere Strategie $T_i \ne R_i$ des Spielers $i$
  und den daraus resultierenden Strategievektor
  $T \coloneq \lparen T_i \mathcomma R_{ \minus i } \rparen$,
  etwa wie in Abbildung \ref{im:global}.
  Um die Kostenänderungen beim Übergang von $R$ nach $T$ zu untersuchen,
  müssen alle Kanten aus
  $G \lparen R \rparen$ und $G \lparen T \rparen$
  betrachtet werden.
  Dazu setzen wir $E \coloneq E_R \cup E_T$ und definieren
  \[
    r_i \lparen e \rparen
    \coloneq
    \symbb{1}_{ R_i } \lparen e \rparen
    \cdotp \frac{c_e}{k_e \lparen R \rparen}
    \quad
    \text{sowie}
    \quad
    t_i \lparen e \rparen
    \coloneq
    \symbb{1}_{ T_i } \lparen e \rparen
    \cdotp \frac{c_e}{k_e \lparen T \rparen}
  \]
  für alle $e \in E$.
  Diese Werte entsprechen genau den Kosten,
  die für Spieler $i$ durch die Kante $e$ unter $R$ bzw.\ $T$ entstehen.
  Folglich gilt
  \begin{equation}
    \label{eq:global_potential1}
    c_i \lparen R \rparen
    \equal
    \sum_{ e \in E } r_i \lparen e \rparen
    \quad
    \text{und}
    \quad
    c_i \lparen T \rparen
    \equal
    \sum_{ e \in E } t_i \lparen e \rparen.
  \end{equation}

  Wir zeigen nun,
  dass für alle $e \in E$ die Gleichung
  \begin{equation}
  \label{eq:global_potential2}
    \mupPsi_e \lparen R \rparen \minus \mupPsi_e \lparen T \rparen
    \equal
    r_i \lparen e \rparen \minus t_i \lparen e \rparen
  \end{equation}
  gilt und erhalten so mit (\ref{eq:global_potential1}) die Behauptung:
  \begin{align*}
    \mupPsi \lparen R \rparen \minus \mupPsi \lparen T \rparen
    &\equal
    \sum_{ e \in E } \mupPsi_e \lparen R \rparen
    \minus \sum_{ e \in E } \mupPsi_e \lparen T \rparen
    \equal
    \sum_{ e \in E }
    \Big \lparen \mupPsi_e \lparen R \rparen
    \minus \mupPsi_e \lparen T \rparen \Big \rparen \\
    &\equal
    \sum_{ e \in E }
    \Big \lparen r_i \lparen e \rparen
    \minus t_i \lparen e \rparen \Big \rparen
    \equal
    \sum_{ e \in E } r_i \lparen e \rparen
    \minus \sum_{ e \in E } t_i \lparen e \rparen
    \equal
    c_i \lparen R \rparen \minus c_i \lparen T \rparen.
  \end{align*}

  Jede Kante $e$ aus
  $R_i \cap T_i$ oder $\lparen R_i \cup T_i \rparen ^\mathrm{C}$
  bleibt von dem betrachteten Strategiewechsel unberührt.
  Daher folgt
  $r_i \lparen e \rparen \equal t_i \lparen e \rparen$
  sowie
  $\mupPsi_e \lparen T \rparen \equal \mupPsi_e \lparen R \rparen$
  und so auch (\ref{eq:global_potential2}).

  Für $e \in R_i \setminus T_i$ ist sowohl
  $r_i \lparen e \rparen \equal \frac{c_e}{k_e \lparen R \rparen}$
  als auch $t_i \lparen e \rparen \equal 0$
  und
  $
    \mupPsi_e \lparen R \rparen
    \equal
    \mupPsi_e \lparen T \rparen \mathplus \frac{c_e}{k_e \lparen R \rparen},
  $
  und für $e \in T_i \setminus R_i$ erhalten wir
  $r_i \lparen e \rparen \equal 0$
  sowie
  $t_i \lparen e \rparen \equal \frac{c_e}{k_e \lparen T \rparen}$
  und
  $
    \mupPsi_e \lparen R \rparen \mathplus \frac{c_e}{k_e \lparen T \rparen}
    \equal
    \mupPsi_e \lparen T \rparen
  $.
  Damit gilt Gleichung (\ref{eq:global_potential2})
  also auch in diesen beiden Fällen.
\end{proof}

\begin{lemma}
\label{la:global_bounds}
  Sei $S$ ein Strategievektor im Global Connection Game.
  Dann gilt
  \[
    \SC{S} \leq \mupPsi \lparen S \rparen \leq H_k \cdotp \SC{S}.
  \]
\end{lemma}

\begin{proof}
  Für alle $e \in E_S$ gilt wegen $1 \leq k_e \leq k$ zunächst
  $c_e H_1 \leq c_e H_{ k_e } \leq c_e H_k$,
  woraus insbesondere
  \begin{equation}
  \label{eq:global_bounds}
    H_1 \cdotp \sum_{ e \in E_S } c_e
    \equal
    \sum_{ e \in E_S } c_e H_1
    \leq
    \sum_{ e \in E_S } c_e H_{ k_e }
    \leq
    \sum_{ e \in E_S } c_e H_k
    \equal
    H_k \cdotp \sum_{ e \in E_S } c_e.
  \end{equation}
  folgt.
  Weil nach Definition $\SC{S} \equal \sum_{ e \in E_S } c_e$
  und $\mupPsi \lparen S \rparen \equal \sum_{ e \in E_S } c_e H_{ k_e }$ ist,
  lässt sich Ungleichung (\ref{eq:global_bounds}) als
  \[
    H_1 \cdotp \SC{S}
    \leq
    \mupPsi \lparen S \rparen
    \leq
    H_k \cdotp \SC{S}
  \]
  schreiben.
  Damit liefert $H_1 \equal 1$ die Behauptung.
\end{proof}

\begin{satz}
  Die $k$-te harmonische Zahl ist eine obere Schranke
  für den Preis der Stabilität des Global Connection Game.
\end{satz}

\begin{proof}
  Nach Lemma \ref{la:global_potential} ist $\mupPsi$
  eine exakte Potentialfunktion für das Global Connection Game.
  Die Behauptung folgt nun aus Lemma \ref{la:global_bounds}
  und Satz \ref{thm:potential_bounds}.
\end{proof}

\bigskip

Abschließend zeigen wir noch,
dass die Anzahl der Spieler eine obere Schranke
für den Preis der Anarchie des Global Connection Game
bezüglich der Gemeinwohlkosten ist.

\begin{satz}
  Die Anzahl der Spieler $k$ ist eine obere Schranke
  für den Preis der Anarchie des Global Connection Game.
\end{satz}

\begin{proof}
  Die Kosten eines Spielers $i$
  lassen sich für jeden beliebigen Strategievektor $S$
  mit Hilfe der Anzahl an Spielern nach unten abschätzen,
  denn für alle $e \in E_S$ gilt $k_e \lparen M \rparen \leq k$ und so auch
  \begin{equation}
  \label{eq:global_pa}
    c_i \lparen S \rparen
    \equal
    \sum_{ e \in S_i } \frac{c_e}{k_e \lparen S \rparen}
    \geq
    \sum_{ e \in S_i } \frac{1}{k} c_e
    \equal
    \frac{1}{k} \sum_{ e \in S_i } c_e.
  \end{equation}

  Sei nun $M$ ein optimaler Strategievektor und $N$ ein Nash-Gleichgewicht.
  Aus der Annahme $\SC{N} \greater k \cdotp \SC{M}$
  erhalten wir definitionsgemäß zunächst
  \[
    \sum_{ i \equal 1 }^k
    \Big \lparen c_i \lparen N \rparen
    \minus k c_i \lparen M \rparen \Big \rparen
    \equal
    \sum_{ i \equal 1 }^k c_i \lparen N \rparen
    \minus k \cdotp \sum_{ i \equal 1 }^k c_i \lparen M \rparen
    \equal
    \SC{N} \minus k \cdotp \SC{M}
    \greater
    0,
  \]
  also existiert ein $i$ mit
  $c_i \lparen N \rparen \minus k c_i \lparen M \rparen \greater 0$.
  Für diesen Spieler ist eine Strategieänderung vorteilhaft,
  denn gemäß Beobachtung (\ref{eq:global_pa}) folgt weiter
  \[
    c_i \lparen N \rparen \minus \sum_{ e \in M_i } c_e
    \equal
    c_i \lparen N \rparen
    \minus k \left \lparen \frac{1}{k} \sum_{ e \in M_i } c_e \right \rparen
    \geq
    c_i \lparen N \rparen \minus k c_i \lparen M \rparen
    \greater
    0,
  \]
  und so insbesondere
  $c_i \lparen N \rparen \greater
  \sum_{ e \in M_i } c_e \geq c_i \lparen M \rparen$.
  Dies steht jedoch im Widerspruch dazu,
  dass~$N$ ein Nash-Gleichgewicht ist.
  Folglich gilt $\SC{N} \leq k \cdotp \SC{M}$
  und weil $N$ beliebig war, ist der Preis der Anarchie also höchstens $k$.
\end{proof}
