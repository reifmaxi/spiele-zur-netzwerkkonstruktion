\section{Fazit}

Der Preis der Stabilität und der Preis der Anarchie
sind zwei wichtige Kennzahlen,
um die Kosten von Nash-Gleichgewichten
mit den mindestens auftretenden Kosten zu vergleichen.
Für Spiele mit exakter Potentialfunktion, wie dem Global Connection Game,
lässt sich der Preis der Stabilität
durch die Potentialfunktionmethode beschränken.

Im Local Connection Game existieren für die meisten Werte von $\mupalpha$
optimale Nash-Gleichgewichte
und auch der Preis der Anarchie lässt sich
in Abhängigkeit von $\mupalpha$ beschränken.
Die Kosten der Kanten haben also
maßgeblichen Einfluss auf das Aussehen von Nash-Gleichgewichten.
Dies bietet dem Spieldesigner die Möglichkeit,
etwa durch Subventionen zu beeinflussen,
wie die entstehenden Netzwerke wahrscheinlich aussehen werden.

Im Global Connection Game sind
sowohl der Preis der Stabilität als auch der Preis der Anarchie
abhängig von der Anzahl der Spieler.
Je mehr Spieler teilnehmen,
desto ineffizienter können Nash-Gleichgewichte werden.
Dem kann etwa durch eine Teilnehmerbeschränkung entgegengewirkt werden.
