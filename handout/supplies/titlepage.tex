\begin{titlepage}
  \maketitle
  \begin{abstract}
    Die vorliegende Arbeit beschäftigt sich
    mit einmaligen simultanen Spielen zur Netzwerkkonstruktion.
    Im Speziellen wird die Existenz von Nash-Gleichgewichten
    im \emph{Local Connection Game}
    und im \emph{Global Connection Game}
    nachgewiesen
    und die jeweils entstehenden Kosten mit Hilfe
    des Preises der Stabilität sowie des Preises der Anarchie
    mit den minimal möglichen Kosten verglichen.
    Die beiden Spiele entstammen dem
    von Éva Tardos und Tom Wexle geschriebenen Kapitel
    \emph{Network Formation Games and the Potential Function Method}
    aus dem Buch \emph{Algorithmic Game Theory} \cite{tardos_wexler_2007}.
  \end{abstract}
  \vfill
  \tableofcontents
  \vfill
  \bigskip
  \bigskip
  \begin{figure}[b]
    \centering
    \includegraphics*[width=0.45\textwidth]{logo}
  \end{figure}
  \thispagestyle{empty}
\end{titlepage}
